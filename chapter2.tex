\chapter{Ring and Field}

\section{Zoom Table}

Abbreviated specification:
\begin{itemize}
    \item \textbf{ID}: Integral domain.
    \item \textbf{CR}: Commutative ring.
    \item \textbf{PID}: Principle ideal domain.
    \item \textbf{UFD}: Unique factorization domain.
    \item \textbf{EID}: Euclidean integral domain.
\end{itemize}

\begin{center}
    \begin{tikzcd}
        a\text{ is prime element} 
        \arrow[rrr, "\bmrm{A1}. \ R\text{ is ID}", shift left] 
        \arrow[dd, "\bmrm{A3}. \ R \text{ is ID}"', leftrightarrow] 
        &  &  &
        a \text{ is irreducible element} 
        \arrow[lll, "\bmrm{A2}.", shift left]
        \arrow[dd, "\bmrm{A5}. \ R \text{ is PID}", shift left] \\
        &  &             \\
        (a) \text{ is prime ideal}
        & & & 
        (a) \text{ is maximal ideal}
        \arrow[lll, "\bmrm{A6}. \ R \text{ is CR}"]
        \arrow[uu, "\bmrm{A4}. \ R \text{ is ID}", shift left]
    \end{tikzcd}
\end{center}
Other conclusions:
\begin{enumerate}
    \item[\textbf{B1}.] $R$ is EID $\Rightarrow$ $R$ is PID.
    \item[\textbf{B2}.] $R$ is PID $\Rightarrow$ $R$ is UFD.
    \item[\textbf{B3}.] (i) $R$ is ID, (ii) every proper factor chain in $R$ is finite, (iii) every irreducible element is prime element $\Rightarrow$ $R$ is UFD. 
    \item[\textbf{B4}.] (i) $R$ is ID, (ii) every $r \in R$ can be denoted as a multiplication of irreducible elements, (iii) every irreducible element is prime element $\Leftrightarrow$ $R$ is UFD.
    \item[\textbf{B5}.] $R$ is UFD $\Rightarrow$ $\exists \gcd(a, b)$.
\end{enumerate}







\noindent{\large\textbf{Proof}}

% \textbf{A1}. Let $p$ be a prime element, suppose $a | p$, then $p | ab$. If $p | a$, we have $p \sim a$. On the other hand, if $p | b$, then $p = hpa$. Due to there is no zero divisor in ID, we conclude that $pa = 1$, which implies $a \sim 1$.


% \textbf{A2}.
% \begin{enumerate}
%     \item[case1.] $R$ is ID, and every two elements in $R$ have the greatest common factor.
% \end{enumerate}
\begin{enumerate}
    \item[\textbf{A1}.] Let $p$ be a prime element, suppose $a | p$, then $p | ab$. If $p | a$, we have $p \sim a$. On the other hand, if $p | b$, then $p = hpa$. Due to there is no zero divisor in ID, we conclude that $ha = 1$, which implies $a \sim 1$.
    

    \item[\textbf{A2}.]
    \begin{enumerate}
        \item[\textbf{case1}.] $R$ is ID, and every two elements in $R$ have the greatest common divisor. The proof proceeds as follows: Let $p$ be the irreducible element in $R$, and $p | bc$. Then $\gcd(p, b)$ is either the invertible element of $R$ or the equivalent element of $b$. If $\gcd(p, b) \sim 1$, then $\gcd(cp, cb) \sim c$ (\cite{Qiu_MA} p.146.). We have $p | \gcd(cp, cb) \land \gcd(cp, cb) | c$, thus $p | c$. If $\gcd(p, b) \sim p$, we immediatly get $p | b$.
        \item[\textbf{case2}.] $R$ is PID. (The proof can be performed as the process that $R$ is PID $\Rightarrow$ $R$ is UFD $\Rightarrow$ $\exists \gcd(a, b)$. The following we provid another way of solution, see \cite{LWW_AJN} Lemma 6.2.9). Suppose $p$ is a prime element and $p|ab$, there exists $f$ such that $\langle p, a \rangle = (f)$. Therefore $(a) \subset (f)$ and $(p) \subset (f)$ hold, which is equivalent to $f|a$ and $f|p$. If $f \sim 1$, then $\langle a, p \rangle = R$, which implies there exists $u, v$ such that $ua + vp = 1$. Thus we have $uab + vpb = b$, hence $p | uab + vpb = b$. If $f \sim p$, we immediatly get $p | a$.
    \end{enumerate}


    \item[\textbf{A3}.] $a$ is a prime $\Leftrightarrow$ $a \neq 0 \land a \notin R^{\times} \land (a|bc \Rightarrow a|b \lor a|c ) \Leftrightarrow (a) \neq (0) \land (a) \neq R \land (bc \in (a) \rightarrow b \in (a) \lor c \in (a))$.
    \item[\textbf{A4}.] By \textbf{A6}. \textbf{A3}. and \textbf{A1}.
    \item[\textbf{A5}.] Suppose $(a) \subset I \subset R$. By prescribed condition that $R$ is PID, so we have $I = (b)$. Thus either $b \sim 1$ or $b \sim a$ holds.
    \item[\textbf{A6}.] $(a)$ is maximal ideal $\Leftrightarrow$ $R/(a)$ is a filed $\Rightarrow$ $R/(a)$ is an ID $\Leftrightarrow$ $(a)$ is a prime element. 
    

    \item[\textbf{B1}.] EZ.
    \item[\textbf{B2}.]
    \begin{enumerate}
        \item[\textbf{step1}.] $R$ is PID, then every ascending chain of ideals in $R$ stops. To be specific, suppose $(I_n)_{n \geq 0}$ is a series of ideals, which satisfies $I_1 \subset I_2 \subset \cdots$. There must exists $n \in \Z_{\geq 0}$ such that $I_n = I_{n+1} = \cdots$. The proof is as follows: Let $I = \bigcup_{n \geq 0} I_n$, it follows that $I$ is an ideal, thus $I = (h)$. It can be verified that $\exists n \in \Z_{\geq 0}$ that $h \in I_n$, and therefore $I \subset I_n$.
        \item[\textbf{step2}.] If $R$ satisfies the ascending chain condition that every ascending chain of ideals in $R$ stops, then $\forall r \in R^*$ can be denoted as a multiplication of irreducible elements. If $r \in R^{\times}$, we agree that $r$ is a multiplication of 0 irreducible element. If $r \notin R^{\times}$, we let $r_0 = r$ and assume that $r$ has no irreducible factorization. It follows that $r$ is not irreducible, or $r = r$ is a irreducible factorization. Thus we have $r_0 = r_1 s_1$ where $r_1, s_1 \nsim r_0$, which implies that $(r_0) \subsetneq (r_1)$ and $(r_0) \subsetneq (s_1)$. By the assumption that $r$ has no irreducible factorization, we conclude that $r_1$ or $s_1$ remains the same property. Suppose $r_1$ has no irreducible factorization, and continue the process. Finally we end up with a strictly ascending chain of ideals $(r_0) \subsetneq (r_1) \subsetneq \cdots $, which contradicts the discussion in step1.
        \item[\textbf{step3}.] By \textbf{A2}. we conclude that in PID every irreducible element is prime element. The uniqueness of decomposition can be easily verified by using Induction.
    \end{enumerate}

    
    \item[\textbf{B3}.] Similar to \textbf{B2}.
    \item[\textbf{B4}.] $(\Rightarrow)$ is similar to  \textbf{B2}. Next we prove the $(\Leftarrow)$ direction (\cite{LWW_AJN} Proposition 6.3.2). Suppose $R$ is UFD , $p \in R$ is irreducible, and $p | ab$ where $a = q_1 \cdots q_m, \ b = r_1\cdots r_n$. Therefore $\frac{ab}{p}$ can be decomposited as $s_1 \cdots s_t$. Thus $q_1 \cdots q_m r_1\cdots r_n = ab = s_1 \cdots s_t p$. By the uniqueness of decomposition, it follows that $p \sim q_i \lor p \sim r_i$. 
    \item[\textbf{B5}.] Suppose $a = \prod_{i \geq 1} p_i^{n_i}, \ b = \prod_{i\geq 1} p_i^{m_i}$. Let $g_0 = \prod_{i \geq 1} p_i^{\min\{ n_i, m_i \}}$. Recall the definition of gcd in PID that $\langle a_1 , \cdots, a_n \rangle = \gcd(a_1, \cdots, a_n) R = gR$. It directs us toward the proof of $g_0 \sim g$. Pursuant to \cite{LWW_AJN} Proposition 2.7.3, we have $g_0|a \land g_0 | b \Leftrightarrow (\forall x \in \langle a, b \rangle \Rightarrow g_0 | a ) \Leftrightarrow g_0 | g$. To prove the reverse direction, notice that $g | a \land g | b$, which implies that $g = \prod_{i\geq 1} p_i^{t_i}$ and $t_i \leq \min\{ n_i, m_i \}$. We conclude that $g | g_0$.
\end{enumerate}



\section{Unique factorization domain}
\begin{definition}
    $R$ is UFD iff (i) each non-zero element in $R$ can be decomposited into the multiplication of finite many irreducible elements; (ii) this decomposition is Unique in the sense of equivalence that $a \sim b \Leftrightarrow a \in bR^{\times}$.
\end{definition}


\begin{corollary}
    $p$ is an irreducible element, then every element in $p R^{\times}$ is irreducible.
\end{corollary}


Define $P = \{ p \in R: \text{irreducible element} \}$. The equivalence relationship was defined as $a \sim b \Leftrightarrow a \in b R^{\times}$. We hope to decompose $P$ into the disjoint union of equivalence classes, and select a subset consisting of representative elements. To archieve this, the first step is choose an function $g \in \prod (P / \sim)$. Then $\Im g$ is what we want.


\begin{proposition}
    Suppose $\Im g = \{ p_a \}_{a \in A}$. $R$ is UFD. Then $\forall a \in R^*$ can be decomposited as:
    \[
        \underbracket{e}_{\in R^{\times}} \prod_{a \in A} \underbracket{p_a^{n_a}}_{\text{fm } n_a \neq 0 } 
    \]
\end{proposition}

\begin{proof}
    By definition of UFD, we obtain that $a = q_1^{n_1} \cdots q_s^{n_s}$, where $q_i \in [p_i]_{\sim}$. Thus $a = e_1^{n_1} \cdots e_s^{n_s} p_1^{n_1} \cdots p_s^{n_s}$.
\end{proof}



Suppose $R$ is a ring, and $a_1, \cdots, a_s \in R$. In principle, the greatest common divisor of $a_1, \cdots, a_s$ are define as an element $d$ which satisfies the following properties: (i) $d$ is a common factor; (ii) for any common factor $l$ it holds that $l | d$. For other spacial rings, the gcd has different definitions. Below is an example in UFD:



\begin{definition}
    $a_1 , \cdots, a_s \in R^*$, each of them can be decomposited to $e_i \prod_{a \in A} p_a^{n_{ai}}$. Then
    \[
        \gcd(a_1, \cdots, a_s) := \prod_{a \in A} p_a^{\min\{ n_{a1}, \cdots, n_{as} \}}
    \]
    (This definition ensures it's output is $1$ when the inputs are mutual prime.)
\end{definition}





Using the newly definition of gcd, it can be verified that:
\begin{proposition}
    If $a_1 , \cdots, a_s \in R^*$, then:
    \begin{enumerate}
        \item $\forall i,\ \gcd(a_1, \cdots, a_s) | a_i$.
        \item $\forall i \ (d | a_i) \Rightarrow d | \gcd(a_1, \cdots, a_s)$.
    \end{enumerate}
    Additionally, if $r \in R$ satisfies the following properties:
    \begin{enumerate}
        \item $\forall i,\ r | a_i$.
        \item $\forall i \ (d | a_i) \Rightarrow d | r$.
    \end{enumerate}
    then $r \in \gcd(a_1, \cdots, a_s) R^{\times}$. In other words, the set that consisting of all gcd of $a_1, \cdots, a_s$ is $\gcd(a_1, \cdots, a_s) R^{\times}$.
\end{proposition}





\begin{corollary}
    Suppose $a = u \prod_{i=1}^s p_i^{n_i}, b = v \prod_{i=1}^{s} p_i^{m_i}$, then
    \[
        a | b \Leftrightarrow \forall i\ ( n_i \leq m_i) 
    \]
\end{corollary}
\begin{proof}
    Without losing generality, suppose for the contradiction that $n_1 > m_1$, and it leads to a contradiction:
    \begin{align*}
        a | b &\Leftrightarrow \exists h \in R\ (ah = b) \\
        & \Rightarrow p_{i}^{m_1} \left( u h p_1^{n_1 - m_1} p_2^{n_2} \cdots p_s^{n_s} - v p_2^{m_2} \cdots p_s^{m_s} \right) = 0\\
        &\Rightarrow u h p_1^{n_1 - m_1} p_2^{n_2} \cdots p_s^{n_s} =  v p_2^{m_2} \cdots p_s^{m_s} \\
        & \text{(By the definition of UFD )} \Rightarrow n_1 = m_1
    \end{align*}

    For the reverse direction, suppose $\forall i \ (n_i \leq m_i)$. It follows that:
    \[
        u \prod_{i=1}^s p_i^{n_i} \cdot u^{-1} v \prod_{i=1}^s p_{i}^{m_1 - n_i} = v \prod_{i=1}^{s} p_i^{m_i}
    \]
\end{proof}


\begin{proposition}
    $R$ is UFD, then each irreducible element is prime element.
\end{proposition}







\textbf{All subsequent} $R$ \textbf{, if not otherwise specified, is UFD}.


\begin{definition}
    $R[X]^* \ni f = a_n X^n + \cdots + a_0$, and $d$ is one of the gcds of $a_n, \cdots, a_0$, then:
    \[
        f \text{is primitive polynomial} \Leftrightarrow d \sim 1
    \]
\end{definition}



\begin{definition}
    $R[X]^* \ni f = a_n X^n + \cdots + a_0$, then:
    \[
        c(f) := \gcd(a_n, \cdots, a_0)
    \]
\end{definition}




\begin{proposition}
    \label{proposition 2.2.10}
    For any $f \in R[X]^*$:
    \begin{enumerate}
        \item $f = df_0$, where $d \in R^*$ and $f_0$ is primitive polynomial.
        \item If $f = d_1 f_1 = d_2 f_2$, then $d_1 \sim d_2 $in $R$ and $f_1 \sim f_2 $ in $R[X]$.
    \end{enumerate}
\end{proposition}
\begin{proof}
    (1)
    \[
        f = c(f) \sum_{i=0}^n \frac{a_i}{c(f)} X^i \land \gcd\left( \frac{a_0}{c(f)}, \cdots, \frac{a_n}{c(f)} \right) = 1
    \]

    (2)
    Suppose
    \[
        a_1\left( \sum_{i=0}^{n} c_i X^i \right) = a_2\left( \sum_{i=0}^{n} d_i X^i \right)
    \]
    where $\frac{a_2}{a_1}  = \frac{p}{q} \land \gcd(p, q) = 1$, and both polynomials are primitive. It is equivalent to:
    \begin{align*}
        & q \left( \sum_{i=0}^{n} c_i X^i \right) =  p \left( \sum_{i=0}^{n} d_i X^i \right) \\
        & \Rightarrow \forall i ,\ q c_i = p d_i \\
        & \Rightarrow \forall i,\ q | d_i \land p | c_i \\
        & \Rightarrow p\sim 1 \land q \sim 1
    \end{align*}
\end{proof}


\begin{proposition}
    $R$ is UFD, and $F$ is its fraction filed, then for $f_1, f_2 \in R[X]^*$, its holds that:
    \[
        f_1 \sim f_2 \text{ on } R[X] \Leftrightarrow  f_1 \sim f_2 \text{ on } F[X]
    \]
\end{proposition}
\begin{proof}
    See \cite{Qiu_MA} p.156. $(\Rightarrow)$: Obvious. $(\Leftarrow)$: similar to Proposition\ref{proposition 2.2.10}.
\end{proof}





\begin{lemma}
    $f_1, f_2 \in R[X]$ are primitive polynomials, then $f_1f_2$ is primitive polynomial.
\end{lemma}
\begin{proof}
    See \cite{LWW_AJN} Lemma 6.9.4.
\end{proof}








